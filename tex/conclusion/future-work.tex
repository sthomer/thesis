\section{Future Work}
\label{section:future-work}

\subsection{Inner Products and Spatial Geometry}
\label{section:inner-products-spatial-geometry}

One of the most interesting proposals of IDyOT theory is that, since the inner product determines the geometry of the space, we can impose different semantics on a space simply by employing a different inner product \citep{wiggins2018creativity}. In this implementation, the only inner product used corresponds to the Frobenius norm (see section \ref{section:frobenius-norm}), so future work is needed to examine how the categorization, interpolation, and abstraction processes implemented here will be effected by choosing inner products that correspond to different semantics of the space.  For instance, there is evidence that humans represent the pitch space as a spiral \citep{deutsch2013psychology} and represent the color space as a spindle \citep{sivik1994color}.  Inner products could be chosen for a semantic space so that its geometry conforms to these semantics.

\subsection{Spectral Projectors and Reification}
\label{section:spectral-projectors-reification}

One of the main problems of this formalism comes from the use of the Fourier transform operator as the method of producing a spectral respresentation in abstraction.  Since tensor rank promotion results in exponentially larger representations of a category, the processing and memory requirements for a given category become huge after only a few levels of abstraction.  If instead a different formalism is used for a spectral representation instead of the Fourier transform, but can avoid tensor rank promotion, then we retain the spectral time-invariance necessary for abstraction without the computational overhead. Specifically, if instead of using a spectral \textit{operator} like the Fourier Transform, a spectral \textit{projector} could be used.  This would mean that the spectral representation of a trajectory would remain in the same rank Hilbert space at each level of abstraction, thereby avoiding the problem of tensor rank promotion.

An exciting corollary of using a projector instead of operator is that since the abstraction of a trajectory would land in the same Hilbert space, but separate conceptual space, a reification process could be defined.  By mapping that spectral representation back onto its subordinate layer, one could examine an reified abstraction of a trajectory in the same space as that trajectory!

%By imperfect analogy, this would be akin to finding that the spectral representation of a full sentence is equivalent to a single word.  Not only would the geometric properties of this relation be interesting to study, the semantic relation between a trajectory and its reification would be enlightening.
