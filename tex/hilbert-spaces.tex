\chapter{Hilbert Spaces}

\begin{quote}
  "Hilbert spaces are the means by which the ordinary experience of Euclidian concepts can be extended meaningfully into the idealized constructions of more complex math."
\end{quote}

\section{Motivation}

Since we need to represent oscillations in a conceptual space, we need a more general notion of what a space is than a finite-dimensional Euclidian space.  The generalization employed here is that of Hilbert spaces, most famously used in quantum mechanics to model wave functions.

Hilbert spaces are characterized by their inner product $\innerproduct$, which can be used to generate an infinite orthonormal sequence $\orthseq$.  Similar to dimensions $(\hat{x}, \hat{y}, \hat{z})$ that characterize three dimensions in Cartesian space, each element in the orthonormal sequence is normalized and orthogonal to each other element, and therefore can be thought of as a dimension. 

Since any function in a Hilbert space can be represented by its Fourier series over an orthonormal sequence, not only can we represent any function, but by employing a different inner product, we generate a different orthonormal sequence and therefore a different representation of that same function. This allows us to have different perspectives of the same "raw" data.  This is analogous to how the coordinates $(1,1)$ representing $(x,y)$ in a 2-dimensional Cartesian space have a different meaning than the coordinates $(1,1)$ representing $(r, \theta)$ in a 2-dimensional radial space.

\section{Complete Inner Product Space}
\begin{itemize}
  \item \textbf{Vector Space:} Dimensions and rules for combining vectors
  \item \textbf{Norm $\|.\|$:} Measure of the size of a vector
  \item \textbf{Inner Product $\innerproduct$:} Defines orthogonality, projections, and angles of vectors
  \item \textbf{Completeness:} Space is big enough to include norm of converging sequences
\end{itemize}

The induced norm $||.||$ is defined in terms of the inner product $\innerproduct$:
\begin{equation}
  \|f\| = \langle f, f \rangle^{1/2} 
\end{equation}
but only if the Cauchy-Schwarz (triangle) inequality holds:
\begin{equation}
|\langle f, g \rangle| \leq \|f\|\|g\| 
\end{equation}

\section{Gram-Schmidt Orthogonalization}
Given an inner product $\innerproduct$ for Hilbert space $\mathcal{H}$, one can generate a complete orthonormal sequence by performing the following procedure:

Find $\{ \varphi_n \}_{n=0}^{N-1} \in \mathcal{H}$ given $\{ f_n \}_{n=0}^{N-1} \in \mathcal{H}$ independent vectors
\begin{equation}
\begin{gathered}
  \varphi_0 \overset{\bigtriangleup}{=} \frac{f_0}{\| f_0 \|} , \quad
  \nu_n \overset{\bigtriangleup}{=} f_n - \sum_{m=0}^{n-1} \langle f_n , \varphi_m \rangle , \quad
  \varphi_n \overset{\bigtriangleup}{=} \frac{\nu_n}{\| \nu_n \|}
\end{gathered}
\end{equation}

Essentially, at each step n, remove all lower $\varphi_n$ projections from the current function $f$ and normalize it.  By removing all projections on lower $\varphi_n$ at each step, we ensure that what results is orthogonal to everything below it. 

\section{Representing Oscillations}

Functions can be thought of as point or vector in a given Hilbert Space characterized by its inner product $\innerproduct$. Since an oscillation is just a periodic function, we can represent any oscillation to full precision as a point in a Hilbert space.  Specifically, any given function $f$ can be decomposed into its Fourier series:

\begin{equation}
  f = \sum_{n=0}^\infty \langle f, \varphi_n \rangle \varphi_n
\end{equation}

Since each $\varphi_n$ represents an orthonormal basis vector, this decomposition allows us to represent any oscillation by an infinite vector, where each dimension of the vector corresponds to a $\varphi_n$.


