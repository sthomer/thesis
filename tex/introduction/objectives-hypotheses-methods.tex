\section{Objectives, Hypotheses, \& Methods}
\label{section:objectives-hypotheses-methods}

The objectives of this thesis are three-fold.  The first goal is to determine how utilizing Hilbert spaces and the Fourier transform can be used to implement the processes of segmentation, categorization, and abstraction described in the IDyOT (section \ref{section:idyot} and chapter 3).  The second is to implement the theoretical results, with the third being to determine if the implemented system behaves as described in the theory when applied to a corpus of human speech audio.

Since the first goal is not so much a matter of empirical testing, but of theoretical explication and derivation, there is no real hypothesis for this objective.  That being said, if those derivations, when implemented in the system, yield poor results, it's possible that instead of the theory being deficient, it is actually the derivations.  Therefore, the main hypothesis of the thesis is that the implemented system will produce results in line with those described in the theory.  Specifically, as applied to audio data of human speech, the hypothesis is that at a few levels of abstraction, the categories corresponding to human speech syllables will emerge.

The research method is split primarily into two categories, though with significant overlap. The first category investigates the consequences of the choice of Hilbert spaces and the Fourier transform as the driving mathematical formalisms in an implementation of IDyOT.  This portion of the thesis is highlighted in Chapter 3: Theory \& Implementation and focuses on how the mathematics of the main processes of IDyOT shake out when using the aforementioned formalisms.  The theoretical results described in this section are then used to create an implementation of the IDyOT system, which is trained on audio data of human speech.  This implementation and analysis of results form the second category of the research method.

The results of training the implementation on human speech are then presented and discussed in Chapters 4 and 5.  Since a quantitative comparison of the cognitive architecture implementation with the human mind is obviously impossible, we instead perform a qualitative analysis of the results.  The visualizations presented in the section \ref{section:results} provide a medium in which the results can be analyzed such that reasonable conclusions can be made.  This is not to say that there are no quantitative techniques employed in the visualizations, just that the analysis of those visualizations is qualitative in manner.
