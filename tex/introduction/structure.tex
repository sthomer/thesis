\section{Structure}
\label{section:structure}

Following this introduction, the structure of the thesis is as follows:

\textbf{Chapter 2: Background:} First, a few short sections of background information regarding conceptual spaces, Hilbert spaces, and information theory will be explained. These are necessary to understand the final background section, Information Dynamics of Thinking, of which this thesis is an implementation.  This section will give the motivations for IDyOT theory as a cognitive architecture and explain its main principles of information dynamics and efficiency.  It  will also outline the main processes of segmentation, categorization, and abstraction core to the functioning of IDyOT.

\textbf{Chapter 3: Theory \& Implementation:} Following the high-level description of IDyOT, more specific matters of theory stemming from the chosen formalisms will be covered.  Explications of utilizing Hilbert spaces and the Fourier transform are explored in the Abstraction and Segmentation sections, whereas the Categorization and Interpolation sections round out the chapter in implementation, tying everything together.

\textbf{Chapter 4: Empirical Analysis:}  The implemented system described in Theory \& Implementation is then empirically tested on a corpus of human speech, with the results visualized to allow for qualitative evaluation against human speech.  These visualizations are explained in the context of IDyOT with the preliminary results suggesting that the different processes of IDyOT are behaving as envisioned in the theory.

\textbf{Chapter 5: Evaluation \& Discussion:} After looking at the specifics of the results in the Empirical Analysis, the overall strengths and deficiencies of both the theory and implementation are investigated in the Discussion, finding that though certain aspects of the results of the implementaion are inconsistent, overall, the implementation at least partially confirms the theory.  After this, the broader implications of IDyOT as a general approach to perceptual representation will be examined.

\textbf{Chapter 6: Conclusion:} Finally, the contributions and limitations of not only the implementation but also the theory are summarized, finishing with prospects for future work that can expand the semantic range of the implementation through the use of different geometries and transformations.
