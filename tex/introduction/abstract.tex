\begin{abstract}

  The Information Dynamics of Thinking (IDyOT) is a cognitive architecture based around the principle that the human mind seeks to represent data in the most information-efficient way possible.  By blending conceptual spaces theory and information theory, and making a minimal set of assumptions, IDyOT specifies three main processes -- segmentation, categorization, and abstraction -- that hierarchically generate spectral representations of perceptual data in an information-efficient manner.

  This thesis is primarily an explication of the propositions and claims introduced in IDyOT by applying the mathematical formalisms of Hilbert spaces and the Fourier transform as implementations of the aforementioned theoretical processes.  After deriving the necessary mathematics in order to implement the architecture, an implemented system is empirically tested on a corpus of human speech, where we expected to see the emergence of distinct semantic categories corresponding to human speech sound syllables.

  After running the implemented system on human speech audio, we found that the hierarchical process of abstracting spectral representations of the audio signals was consistent with a few claims of the theory.  First, spectral representations of similar trajectories through conceptual spaces were categorized together, indicating that hierarchical abstraction produces consistent categories as expected.  Second, the duration of categories at higher levels demarcated boundaries that approximiately corresponded to annotated boundaries of syllables, indicating the segmentation process finds reasonable boundaries at which to chunk information.  Finally, we saw the emergence of similar categories representing distinct human syllables at higher levels of abstraction, indicating that the spectral representations produced at these higher levels have semantic content.  These results provide the first empirical evidence that the IDyOT cognitive architecture is a potentially viable model of human cognition.

\end{abstract}
