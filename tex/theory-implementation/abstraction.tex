\section{Abstraction}

When a segment is produced in the segmentation process, it is abstracted to a single symbol in the superior abstraction layer.  Since each dimension is composed of both an alphabet of symbols and a conceptual space in which those symbols live, a segment of symbols can be though of as a trajectory through a conceptual space.  Viewing this trajectory as a sequence of points in a high-dimensional space, one can abstract out the time variant properties of the signal by viewing it in terms of its spectral properties.  When viewed spectrally, this same trajectory can be thought of as a single point, creating a mapping from the segment in the inferior dimension to the a point in the superior abstraction layer.

\subsection{Fourier Transform}
In the Information Dynamics of Thinking, the Fourier Transform is used to take the time-domain trajectory of the inferior layer to the frequency-domain point of the superior layer.  Though there are other spectral representations of a signal available besides the Fourier transform, there is evidence, especially in the auditory domain, that the brain operates on frequency transformations of time-varying signals (organ of corti), and so it is employed here.

Since the segment is a trajectory of discrete points in a Hilbert space, we use the discrete fourier transform to take the trajectory from time-domain to frequency-domain.  It is important to note that the (discrete) Fourier transform is a linear operator on the Hilbert space, meaning essentially that the \textit{shape} of the signal does not change in the tranformation, only its domain.  Therefore, use the independence of the frequency bins resultant from the DFT to represent those bins as dimensions in the superior abstraction layer.

\subsection{Tensor Rank Promotion}
To clarify this point, consider figure (below).  When the time-varying sound signal is transduced at the base perceptual level, it can be represented as a row vector of real values, where each element in the row represents the signal amplitude dimension at that time step. This sounds signal is of finite length, and can be thought of as a moving window of attention. Since it is a linear operator, performing the DFT on this signal results in a row vector of the same length, only now with complex elements in the frequency domain.  Since each element of this row vector represents an orthonormal frequency bin, we can alternatively think of this row vector as a column vector, where each row in the column represents a different dimension.  This transposition of row vector to column vector is what takes the frequency-domain trajectory to a single point in its dual space.

Now, when we have a trajectory of in this frequency space, instead of operating on a row of real values as in the sound signal, we are operating on a row of complex column vectors, and the trajectory now looks like a matrix.  Taking the DFT of this trajectory and transposing it to be represented as a point, the abstracted point is now a matrix.  Taking a trajectory of these matrices results in a point represented by a cube (3d rectangle?).  Taking a trajectory of these cubes results in a point represented by a hypercube, and so on.

Therefore, when moving from one abstraction layer to its superior, the complex tensor representing the contents of the symbol increases in rank, which is what is meant by tensor rank promotion.

(Figure of tensor rank promotion)

\subsection{Component-wise Fourier Transform}
When performing the Fourier transform on a tensor of any rank, it should be noted that each component in the tensor is independent from every other component.  Though this is not necessarily true in general, due to the particular hierarchical construction of these spaces, each component is decoupled from the rest.  Starting at the bottom, the time-domain sound signal is transformed into a frequency-domain signal of coefficients in independent frequency bins.  It is this independence that allowed us to represent it as a point with those frequency bins as dimensions.  Performing the DFT on the trajectory of column vectors results in independent frequency bins filled with column vectors with independent entries, meaning all components are independent one another.

Recursively performing the DFT on tensors with independent components results in higher rank tesnors with independent components.  This component-independence of the tensor means that the DFT should be taken component-wise, since all other cross-component terms would involve orthgonal components, resulting in zero terms.
