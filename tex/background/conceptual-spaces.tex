\section{Conceptual Spaces}
\label{section:conceptual-spaces}

In knowledge representation, the argument over the representation of cognition often falls into two camps: connectionism and symbolicism.  As the lowest level of represention, connectionism is best exemplified by artificial neural networks \citep{lecun2015deep}, where cognition emerges from myrid connections between neurons.  At the highest level, symbolicism views the mind as a Turing machine \citep{turing2009computing}, where cognition is equivalent to compution by symbol manipulation.  Conceptual spaces theory \citep{gardenfors2004conceptual} argues for a middle way, through the use of eponymous conceptual spaces.  Conceptual spaces theory views cognition as the process of concept formation by means of similarity, so that the continuous representations of connectionism can be bridged to the discrete representations of symbolicism, hopefully gaining the best of both worlds.  

Conceptual spaces theory states that representations in the mind are situated in conceptual spaces -- semantically rich spaces with geometric properties -- which allow for intuitive geometric reasoning about related objects. For instance, in the conceptual space of color with dimensions of hue, saturation, and brightness, one can formally make a geometric claim that “orange lies between yellow and red.”  This simple claim cannot be made by connectionism or symbolicism without imposing ad-hoc external semantics on the connection weights or symbols respectively, highlighting the explanatory power of conceptual spaces for knowledge representation.

\subsection{Quality Dimensions}
\label{section:quality-dimensions}

Quality dimensions are the basic building blocks of a conceptual space.  They can be thought of as the axes that give meaning to the elements in the space.  In three-dimensional Cartesian space, when referring to a point, we specify it by its placement on each of the $x$, $y$, and $z$ -axes.  By analogy, each of the $xyz$-axes would be a quality dimension in the 3D Cartesian space.  However, quality dimensions are more than just orthogonal unit vectors, they can also have their own specific geometry that serves to constrain the dimension. For instance, a quality dimension may have the geometry of a circle, resulting in different behavior than the real number line.  Quality dimensions also allow us to speak meaningfully about similarity between objects in a space since, by definition, they possess distance and betweenness relations.  Finally, it is important to remember that the 'quality' of the quality dimension is what gives it inherent semantic content beyond just being a descriptive dimension.

\subsection{Domains and Conceptual Spaces}
\label{section:domains-and-conceptual-spaces}

Integral quality dimensions require one another to exist.  For instance, the three qualities of sound: pitch, timbre, and loudness, are all integral to one another \citep{mcadams1985qualities}.  It is impossible to identify a sound without specifying all three of these dimensions.  On the other hand, most quality dimensions are separable, meaning that they are independent of one another.  Though separable quality dimensions are independent, they may still be highly correlated, which may give the illusion that they are integral.

A domain is a set of integral dimensions that are separable from all other dimensions.  In a sense, it is the minimum description needed for a given space.  For example, the three qualities of sound form a domain.  Often, different domains will be correlated with one another, and combining them will yield a richer description of a given object.  This combination of multiple domains is what is referred to as a conceptual space, so that the specification of an object is nothing else but its location in a conceptual space.

\subsection{Similarity, Distance, Betweenness}
\label{section:similarity-distance-betweenness}

Humans have an innate sense of similarity without being able to fully describe why two things are similar \citep{tversky1977features}.  In simple cases, this similarity can be made formally explicit, for example that a rectangle is more similar to a square than a circle.  However, this intuition for similarity extends to even very abstract realms.  For instance, most people would naturally agree that country music is closer to rock-n-roll than it is to classical Indian ragas, but would be hard-pressed to give an exact definition or method of why this is so.

Conceptual spaces allow this innate sense of similarity to be codified in the intuition of geometry, betweenness, and distance.  Given a betweenness relation for a conceptual space, one can say that a given object is between two other objects, which allows us to say that one is closer to another than the third.  In our case, we will study conceptual spaces equipped with a distance measure or norm.  This allows us to examine similarity between objects, such that more similar onjects will be closer to one another than more different objects.

\subsection{Convexity of Properties and Concepts}
\label{section:convexity-properties-concepts}

Given that Conceptual Spaces Theory posits that cognition is equivalent to the formation of concepts \citep{gardenfors2004conceptual}, one should define a concept.  Defining a property or concept as "an invariance across a range of contexts, [reifiable] so that it can be combined with other appropriate invariances" \citep{kirsh1991today}, it is immediately clear that these correspond to regions of a conceptual space.  Since all of the objects in a given region are similar to each other, by grouping them together, we can see that the region corresponds to a property or concept. A property would be a region in a domain -- for instance the red property corresponds to a region of the color space -- and a concept would be a region in a conceptual space, with the property or concept being identified by something like its centroidal prototype \citep{rosch1983prototype}.

G{\"a}rdenfors posits that regions corresponding to properties and concepts are convex in nature \citep{gardenfors2004conceptual}.  Though this does not fall directly from the theory itself, it is reasonable to think that the region of concept is not intruded upon by other concepts.  For example, in the color space, the property of red is convex, since we don't see another color like blue interloping into the red region, which would appear as a small area of blue surrounded by a region of red.

\subsection{Higher-order Conceptual Spaces}
\label{section:higher-order-conceptual-spaces}

Higher-order conceptual spaces can be created from combinations and transformations of one or more lower-order conceptual spaces \citep{chella2008cognitive}.  The key notion of the higher-order nature of these spaces is that they are more abstract than the spaces they are generated from.  This is to distinguish higher-order conceptual spaces from combinations that serve to more tightly constrain a space, similar to the intersection set operation.  For instance, overlaying the full color space over the space of human phenotypes results in a restricted color space of human skin tones, not a more abstract space.

Since by nature, quality dimensions and domains often describe low-level quantities like color and sound, it is necessary to combine them into higher-level, more abstract spaces possessing more explanatory power.  Intuitively, the higher level a given conceptual space is, the richer its semantics, so to arrive at a space with sufficient descriptive capabilities for a given cognitive representation, it may be necessary to recursively abstract conceptual spaces into higher-order spaces to arrive at something nontrivial with interesting semantics.
