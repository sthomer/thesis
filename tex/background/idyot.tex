\section{Information Dynamics of Thinking}
\label{section:idyot}

Cognitive architecture blending information theory and conceptual spaces.

\subsection{Boundary Entropy Segmentation}
\label{subsection:boundary-entropy-segmentation}

In natural language processing, n-gram models are often employed to track the frequency of chains of symbols in a given signal.  For instance, if examining textual data, a unigram model would count the frequency of individual words, whereas a bigram model would track the frequency of pairs of words.  It is clear that one can then utilize the information content measure on a unigram model and the conditional entropy measure on the bigram model in order to quantify the amount of information in that model. \cite{sproat1996stochastic}

\subsection{Sequential Memory}
\label{subsection:sequential-memory}

The sequential memory in IDyOT is represented as a hierarchy of chains of symbols. As the agent perceives a continuous stream of perception from the environment, it first discretizes that stream into moments, and links those discretized percepts as symbols in an ever-growing chain.  By examining time-varying information-theoretic properties of this chain, it can be chunked into a series of segments composed of a sequence of symbols.  This segment can then be abstracted by a representative symbol in the superior layer and can be said to subsume the inferior segment.  This process is done recursively until no more abstraction can be performed.

\subsection{Spectral Representations of Meaning}
\label{subsection:semantic-spectral-representation}

\subsection{Semantic Memory}
\label{subsection:semantic-memory}

If the sequential memory can be though of as how concepts relate to eachother over time, semantic memory can be thought of as how those concepts relate to eachother outside of time.  In IDyOT this is modeled using conceptual spaces.  At any given layer of abstraction in the sequential memory, there is a "parallel" semantic space for the symbols of that layer.  This allow us to think of a segment of symbols in sequential memory as a trajectory of concepts in semantic memory.  By looking at a spectral representation of this trajectory, the time-varying properties of the segment are removed, and this spectral representation can be thought of as an abstraction of the segment, and therefore as the abstracted symbol in the superior layer of the sequential memory.
